%! Author = csefalvayk
%! Date = 7/4/21
\documentclass{article}
\usepackage{amsfonts}
\usepackage{url}


% Bibliography styling
\usepackage[super,square,sort&compress,numbers]{natbib}
\bibliographystyle{unsrtnat}
\usepackage{graphicx}
\usepackage{hyperref}

\title{VAERS data reveals no increased risk of neuroautoimmune adverse events from COVID-19 vaccines}

% Authors, for the paper (add full first names)
\author{Chris von Csefalvay\thanks{Starschema Inc., Arlington, VA. Correspondence: \texttt{csefalvayk@starschema.net}.}}

\begin{document}

\maketitle

\begin{abstract}
TBD.
\end{abstract}

\section{Introduction}

Vaccines are, by their nature, intended to elicit immunogenic effects in the recipient.
Where the immune response results in an off-target humoral or cellular immune reaction against the recipient's own tissue, the clinical picture is typically an autoimmune pathology.
The notion of "vaccine induced autoimmunity", which altogether overstates how rare and marginal such reactions are, nonetheless numbers among the key controversies in the fight against vaccine-preventable diseases.\cite{10.1093/cid/ciz135}
One must look no further than the story of the ill-fated recombinant hepatitis B vaccine (HBV) for a case study of the devastating effects that may result when the suspicion of inducing autoimmunity is cast upon a vaccine.
Despite convincing evidence to the contrary,\cite{doi:10.1056/NEJM200102013440502} a number of cases of multiple sclerosis have been attributed to HBV in the late 1990s.
Factual evidence for the lack for such an association notwithstanding,\cite{DeStefano_2002} the effects in the public consciousness endure.
Eurosurveillance reports that in France, where the public discourse about the HBV-MS association was perhaps the most intense, hepatitis B vaccination uptake was consistently below 50\% in all segments of the population\cite{Rey_2018} – a long way from the recommendation of universal hepatitis B vaccination.

The endeavour to reach collective immunity to COVID-19 by way of vaccination is, in this sense, no less susceptible to the danger of vaccine hesitancy.\cite{dror2020vaccine}
Qualitative studies attest that autoimmunity is among the key concerns of those refusing vaccination, even among allied health professionals.\cite{berry2021lessons}
This is exacerbated by the dramatic rise in misinformation about the safety of the COVID-19 vaccines,\cite{islam2021covid} especially via social media platforms like TikTok\cite{basch2021global} and Twitter.\cite{kearney2020twitter,krittanawong2020misinformation}
In addition, two of the three vaccines approved in the United States for COVID-19 prophylaxis at the time of writing use mRNA technology to create the SARS-CoV-2 Spike protein antigen in the recipient's body.
The two mRNA vaccines, widely known as the Pfizer/BioNTech and the Moderna vaccines (BNT162b2, Pfizer, Inc., Collegeville, PA, USA and mRNA-1273/elasomeran, ModernaTX, Inc., Cambridge, MA, USA, respectively), are the first prophylactic vaccines to be approved by the Food and Drug Administration that rely on messenger RNA.
Both the 'novelty factor', the emergency circumstances of authorisation and misunderstandings about the mechanism of action have contributed to complex public sentiment on the matter.

The purpose of this paper is to provide a degree of resolution to that complexity by examining early information on autoimmune aetiologies submitted to the Vaccine Adverse Event Reporting System (VAERS) maintained jointly by the CDC and the FDA.
At the time of writing, over 330m doses of the COVID-19 vaccines have been administered in the United States, with a little under half of the population being fully vaccinated.
The speed and effectiveness of this ambitious vaccination project is no doubt historical, but whether it reaches the principal aim of attaining or approaching herd immunity depends largely on the next few months.
By quantitatively assessing the true risk of autoimmune adverse effects on the basis of passive reporting to VAERS, this paper contributes to dispelling concerns about off-target immunogenic effects and autoimmune presentations resulting from the COVID-19 vaccines.

\section{Materials and methods}

\subsection{Data sources}


\subsection{Identifying cases}
	
\subsection{Identifying controls}
	
\subsection{Case-control matching}
	
\subsection{Statistical analysis}
	
\section{Results}

\section{Discussion}

\subsection{Limitations}


\section{Conclusion}

important to cite no-findings see 'thje risk of vaccination-the importance of negative studies'

% Document
\begin{document}


%%%%%%%%%%%%%%%%%%%%%%%%%%%%%%%%%%%%%%%%%%
\vspace{6pt}

\section*{Funding}

This research was funded by Starschema Inc. under its intramural research funding programme.

\section*{Data availability}

VAERS reporting data is available from the CDC's website at \url{https://vaers.hhs.gov}.
All code and scripts supporting this manuscript are deposited at
\url{https://github.com/chrisvoncsefalvay/covid-19-autommune-aefis} and are made available under the DOI 10.5281/XXXXXXX.

\section*{Conflicts of interest}

CvC is a consultant to a company that may be affected by the research reported in this paper.
The funders had no role in the design of the study;
in the collection, analysis, or interpretation of data;
in the writing of the manuscript, or in the decision to publish the~results.


\section*{References}

\bibliography{bibliography}

\end{document}
