%! Author = csefalvayk
%! Date = 7/4/21
\documentclass{article}
\usepackage{amsfonts}
\usepackage{url}


% Bibliography styling
\usepackage[super,square,sort&compress,numbers]{natbib}
\bibliographystyle{unsrtnat}
\usepackage{graphicx}
\usepackage{hyperref}

\title{A case-control study of autoimmune AEFIs following COVID-19 vaccination reported to VAERS}

% Authors, for the paper (add full first names)
\author{Chris von Csefalvay\thanks{Starschema Inc., Arlington, VA. Correspondence: \texttt{csefalvayk@starschema.net}.}}

\begin{document}

\maketitle

\begin{abstract}
TBD.
\end{abstract}

\section{Introduction}

Vaccines are, by their nature, intended to elicit immunogenic effects in the recipient.
Where the immune response results in an off-target humoral or cellular immune reaction against the recipient's own tissue, the clinical picture is typically an autoimmune pathology.
The notion of "vaccine induced autoimmunity", which altogether overstates how rare and marginal such reactions are, nonetheless numbers among the key controversies in the fight against vaccine-preventable diseases.\cite{10.1093/cid/ciz135}
One must look no further than the story of the ill-fated recombinant hepatitis B vaccine (HBV) for a case study of the devastating effects that may result when the suspicion of inducing autoimmunity is cast upon a vaccine.
Despite convincing evidence to the contrary,\cite{doi:10.1056/NEJM200102013440502} a number of cases of multiple sclerosis have been attributed to HBV in the late 1990s.
Factual evidence for the lack for such an association notwithstanding,\cite{DeStefano_2002} the effects in the public consciousness endure.
Eurosurveillance reports that in France, where the public discourse about the HBV-MS association was perhaps the most intense, hepatitis B vaccination uptake was consistently below 50\% in all segments of the population\cite{Rey_2018} – a long way from the recommendation of universal hepatitis B vaccination.

The endeavour to reach collective immunity to COVID-19 by way of vaccination is, in this sense, no less susceptible to the danger of vaccine hesitancy.\cite{dror2020vaccine}
Qualitative studies attest that autoimmunity is among the key concerns of those refusing vaccination, even among allied health professionals.\cite{berry2021lessons}
This is exacerbated by the dramatic rise in misinformation about the safety of the COVID-19 vaccines,\cite{islam2021covid} especially via social media platforms like TikTok\cite{basch2021global} and Twitter.\cite{kearney2020twitter,krittanawong2020misinformation}
In addition, two of the three vaccines approved in the United States for COVID-19 prophylaxis at the time of writing use mRNA technology to create the SARS-CoV-2 Spike protein antigen in the recipient's body.
The two mRNA vaccines, widely known as the Pfizer/BioNTech and the Moderna vaccines (BNT162b2, Pfizer, Inc., Collegeville, PA, USA and mRNA-1273/elasomeran, ModernaTX, Inc., Cambridge, MA, USA, respectively), are the first prophylactic vaccines to be approved by the Food and Drug Administration that rely on messenger RNA.
Both the 'novelty factor', the emergency circumstances of authorisation and misunderstandings about the mechanism of action have contributed to complex public sentiment on the matter.

The purpose of this paper is to provide a degree of resolution to that complexity by examining early information on autoimmune aetiologies submitted to the Vaccine Adverse Event Reporting System (VAERS) maintained jointly by the CDC and the FDA.
At the time of writing, over 330m doses of the COVID-19 vaccines have been administered in the United States, with a little under half of the population being fully vaccinated.
The speed and effectiveness of this ambitious vaccination project is no doubt historical, but whether it reaches the principal aim of attaining or approaching herd immunity depends largely on the next few months.
By quantitatively assessing the true risk of autoimmune adverse effects on the basis of passive reporting to VAERS, this paper contributes to dispelling concerns about off-target immunogenic effects and autoimmune presentations resulting from the COVID-19 vaccines.

\section{Materials and methods}

\subsection{Data sources}

Reporting data was obtained from VAERS via \url{vaers.hhs.gov} on 04 July 2021, with data including reports up to 02 July 2021.
In order to ensure a sufficient number of controls, data was also obtained going back to 01 January 1995, inclusive.

Maintained jointly by the CDC and the FDA, VAERS is a passive reporting system for potentially vaccine-related adverse events.\cite{chen1994vaccine}
As a passive reporting system, it relies on healthcare professionals, administrators and recipients of vaccines to submit information about potential adverse events following immunisations (AEFIs).
Reports may be made anonymously, and there is no verification of reports against other objective sources, such as death certificates for alleged fatal AEFIs or against hospital EMRs for claimed admissions.
Nor is there an established system to deduplicate reports that may have been submitted by multiple parties to the same event, potentially without each other's knowledge.\cite{von2021vaers}
This results in a significant potential for overreporting of AEFIs to VAERS.

At the same time, there is no general obligation to report most AEFIs to VAERS.
An exception under this are circumstances covered by the National Childhood Vaccine Injury Act of 1986 (42 U.S.C. \S 300aa-1 to 300aa-34).
Each of the Emergency Use Authorizations under which the three COVID-19 vaccines were authorised in the United States mandate reporting of certain AEFIs, specifically

\begin{itemize}
    \item any and all administration errors,
    \item serious adverse effects (AEFIs that result in death, life-threatening illness, inpatient hospitalisation or prolongation of pre-existing hospitalisation, persistent disability or which are determined to be serious in medical judgment),
    \item instances of Multisystem Inflammatory Syndrome, and
    \item cases of COVID-19 necessitating hospitalisation or resulting in death.
\end{itemize}

As a passive reporting system, even with this increased degree of mandatory reporting for COVID-19 vaccines, some AEFIs will go unreported.
The very extent of this underreporting is hard to ascertain.\cite{SINGLETON19992908}
Therefore, the study design was devised so as to mitigate the effects of under- and overreporting as long as these are uniform across vaccines.


\subsection{Identifying cases}

A report was considered a case if at least one of the entries in the reported \texttt{SYMPTOM} fields belonged to the set of relevant diagnoses.
A diagnosis was considered relevant if its MedDRA code fell within the MedDRA High Level Group Terms (HLGT) for autoimmune disorders (10003816) or immune disorders NEC (10027665), but specifically exclude sarcoidoses (10039487), amyloidoses (10002023), blood isoimmune reactions (10023053) and transplant rejection (10052779).
In addition, six diagnoses were explicitly excluded:
ARDS (10001052) and Systemic Immune Response Syndrome (SIRS) (10051379), which are insufficiently specific to an autoimmune cause in the COVID-19 context,
skin sensitisation (10040785) as it is principally of an allergic rather than autoimmune etiology,
shoulder injury related to vaccine administration (10081038) as it permits a wide non-autoimmune etiology, and
the general categories of "adverse event following immunisation" (10069520) and "vaccination complication" (10046861), which were insufficiently specific.
This yielded the following hierarchy (included diagnoses in bold type):

\begin{itemize}
    \item Immune system disorders (10021428) \begin{itemize}
              \item Autoimmune disorders (10003816) \begin{itemize}
                                                        \item \textbf{Autoimmune disorders NEC (10027657)}
                                                        \item \textbf{Blood autoimmune disorders (10003817)}
                                                        \item \textbf{Endocrine autoimmune disorders (10003818)}
                                                        \item \textbf{Hepatic autoimmune disorders (10003820)}
                                                        \item \textbf{Lupus erythematosus and associated conditions (10025136)}
                                                        \item \textbf{Muscular autoimmune disorders (10003821)}
                                                        \item \textbf{Nervous system autoimmune disorders (10074484)}
                                                        \item \textbf{Rheumatoid arthritis and associated conditions (10039075)}
                                                        \item \textbf{Scleroderma and associated disorders (10039711)}
                                                        \item \textbf{Skin autoimmune disorders NEC (10052738)}
              \end{itemize}
              \item Immune disorders NEC (10027665) \begin{itemize}
                                                        \item Acute and chronic sarcoidosis (10039487)
                                                        \item Amyloidoses (10002023)
                                                        \item \textbf{Autoinflammatory diseases (10073080)}
                                                        \item Blood isoimmune reactions (10023053)
                                                        \item \textbf{Immune and associated conditions NEC (10027682)}, except: \begin{itemize}
                                                                                                                                    \item ARDS (10001052),
                                                                                                                                    \item Adverse event following immunisation (10069520),
                                                                                                                                    \item Shoulder injury related to vaccine administration (10081038),
                                                                                                                                    \item Vaccination complication (10046861),
                                                                                                                                    \item Skin sensitisation (10040785), and
                                                                                                                                    \item Systemic Immune Response Syndrome (SIRS) (10051379).
                                                        \end{itemize}
                                                        \item Transplant rejection (10052779)
                                                        \item \textbf{Vasculitides (10052779)}
              \end{itemize}
    \end{itemize}
\end{itemize}

\subsection{Case-control matching}


	
\subsection{Statistical analysis}
	
\section{Results}

\section{Discussion}

\subsection{Limitations}


\section{Conclusion}

important to cite no-findings see 'thje risk of vaccination-the importance of negative studies'


%%%%%%%%%%%%%%%%%%%%%%%%%%%%%%%%%%%%%%%%%%
\vspace{6pt}

\section*{Funding}

This research was funded by Starschema Inc. under its intramural research funding programme.

\section*{Data availability}

VAERS reporting data is available from the CDC's website at \url{https://vaers.hhs.gov}.
All code and scripts supporting this manuscript are deposited at
\url{https://github.com/chrisvoncsefalvay/covid-19-autommune-aefis} and are made available under the DOI 10.5281/XXXXXXX.

\section*{Conflicts of interest}

CvC is a consultant to a company that may be affected by the research reported in this paper.
The funders had no role in the design of the study;
in the collection, analysis, or interpretation of data;
in the writing of the manuscript, or in the decision to publish the~results.


\section*{References}

\bibliography{bibliography}

\end{document}
